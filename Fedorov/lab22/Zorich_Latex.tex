\documentclass[a4paper, 10pt]{book}
\usepackage[english, russian]{babel}
\usepackage[utf8]{inputenc}
\usepackage{amssymb}
\usepackage{amsmath}
\usepackage{amsfonts}
\usepackage{mathrsfs}
\usepackage[left=10mm, top=15mm,
            right=20mm, bottom=15mm,
            nohead, nofoot{geometry}

\setlength{\headheight}{0mm}
\setlength{\headsep}{0mm}
\setcounter{page}{341}

\begin{document}
    \begin{center}
        $\S3.$ Интеграл и производная
    \end{center}

    
    $
    \par
    a)
    \displaystyle
    \int\limits_{-\pi}^{\pi} \sin{mx}\cos{nx} \,dx = \frac{1}{2} \int\limits_{-\pi}^{\pi} (\sin{(n + m)x}-\sin{(n - m)x}) \,dx =
    \\
    \null\hfill=\frac{1}{2} \left( -\frac{1}{n + m} \cos{(n + m)x} + \frac{1}{n - m} \cos{(n - m)x} \right)\Biggr|_{-\pi}^{\pi} = 0,
    $


    \par\noindent
    если $n - m \neq 0$. Случай, когда $n - m = 0$, можно рассмотреть отдельно, и в этом случае, очевидно, вновь приходим к тому же результату.


    $
    \par
    b)
    \displaystyle
    \int\limits_{-\pi}^{\pi} \sin^2{mx} \,dx = \frac{1}{2} \int\limits_{-\pi}^{\pi} (1 - \cos{2mx}) \,dx = \frac{1}{2} \left(x - \frac{1}{2m}sin{2mx} \right)\Biggr|_{-\pi}^{\pi} = \pi.
    $


    $
    \par
    c)
    \displaystyle
    \int\limits_{-\pi}^{\pi} \cos^2{nx} \,dx = \frac{1}{2} \int\limits_{-\pi}^{\pi} (1 + \cos{2nx}) \,dx = \frac{1}{2} \left(x + \frac{1}{2n}sin{2nx} \right)\Biggr|_{-\pi}^{\pi} = \pi.
    $


    \par
    Пример 3.
    \displaystyle
    Пусть $\textit{f} \in \mathscr{R}[-a, a]$. Покажем, что
    \[
    \displaystyle
    \int\limits_{-a}^{a} f(x) \,dx =
    \begin{cases}
    \displaystyle
      2\int\limits_{0}^{a} f(x) \,dx, & \text{если $f$ - четная функция,} \\
      0, & \text{если $f$ - нечетная функция.}
    \end{cases}
    \]


    \par
    Если $f(-x) = f(x)$, то
    \\

    
    $
    \displaystyle
    \int\limits_{-a}^{a} f(x) \,dx = 
    \int\limits_{-a}^{0} f(x) \,dx + \int\limits_{0}^{a} f(x) \,dx =
    \int\limits_{a}^{0} f(-t)(-1) \,dt + \int\limits_{0}^{a} f(x) \,dx =
    \\
    \null\hfill=\int\limits_{0}^{a} f(-t) \,dt + \int\limits_{0}^{a} f(x) \,dx =
    \int\limits_{0}^{a} (f(-x)+f(x)) \,dx = 2\int\limits_{0}^{a} f(x) \,dx
    $

    
    \par
    Если $f(-x) = -f(x)$, то, как видно из тех же выкладок, получим
    \[
    \int\limits_{-a}^{a} f(x) \,dx =
    \int\limits_{0}^{a} (f(-x) + f(x)) \,dx =
    \int\limits_{0}^{a} 0 \,dx = 0.
    \]


    \par
    Пример 4. Пусть \textit{f} - определенная на всей числовой прямой $\mathbb{R}$ периоди-
    ческая функция с периодом \textit{T}, т.е. $f(x+T)=f(x)$ при $x \in \mathbb{R}$.


    \par
    Если $f$—интегрируемая на каждом конечном отрезке функция, то при любом $a \in \mathbb{R}$ имеет место равенство
    \[
    \int\limits_{a}^{a+T} f(x) \,dx = \int\limits_{0}^{T} f(x) \,dx,
    \]
    т.е. интеграл от периодической функции по отрезку длины периода T этой функции не зависит от положения отрезка интегрирования на числовой

    \newpage
    \begin{center}
        ГЛ. VI. ИНТЕГРАЛ
    \end{center}

    \par\noindent
    прямой:

    
    $
    \displaystyle
    \int\limits_{a}^{a+T} f(x) \,dx = \int\limits_{a}^{0} f(x) \,dx + \int\limits_{0}^{T} f(x) \,dx + \int\limits_{T}^{a+T} f(x) \,dx = 
    \\
    \null\hfill =\int\limits_{0}^{T} f(x) \,dx + \int\limits_{a}^{0} f(x) \,dx + \int\limits_{0}^{a} f(t + T)\cdot1 \,dt=
    \\
    \null\hfill =\int\limits_{0}^{T} f(x) \,dx + \int\limits_{a}^{0} f(x) \,dx + \int\limits_{0}^{a} f(t) \,dt =\int\limits_{0}^{T} f(x) \,dx.
    $


    \par
    Мы сделали замену $x = t + T$ и воспользовались периодичностью фун-
    кции f(x).

    \par
    Пример 5. Пусть нам нужно вычислить интеграл $\int\limits_{0}^{1} sin{x^2} \,dx$, например, с точностью до $10^{-2}$.

    \par
    Мы знаем, что первообразная $\int sin{x^2} \,dx$ (интеграл Френеля) не выражается в элементарных функциях, поэтому использовать формулу Ньютона—
    Лейбница здесь в традиционном смысле нельзя.

    \par
    Поступим иначе. Исследуя в дифференциальном исчислении формулу Тейлора, мы в качестве примера (см. гл. V, \S 3, пример 11) нашли, что на отрезке $[-1, 1]$ с точностью до $10^{-3}$ имеет место равенство
    \[
    \sin{x} \approx x - \frac{1}{3!}x^{3} + \frac{1}{5!}x^{5}=:P(x).
    \]

    \par
    Но если $|\sin{x} - P(x)| < 10^{-3} $ на отрезке $[-1, 1]$, то верно также, что
    

\end{document}
